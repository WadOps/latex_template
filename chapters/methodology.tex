\section{Methodology}
In this chapter we present our approach for evaluating the usability of the three end-to-end encryption technologies examined in this project. We decided to follow a threefold approach:
\begin{enumerate}
	\item We identify the most commonly \acrlong{mua} (\acrshort{mua}) also known as email programs, that - natively or by additional plugins -  support at least one of the three technologies \acrshort{pgp}, \acrshort{smime} or \acrshort{pep}. We assess the usability of the encryption features in each of these mail programs \textbf{ourselves}, to get a personal impression as well as to anticipate the challenges that other users might face when cryptographically securing their emails.
	\item We prepare, execute and evaluate an \textbf{online survey}, which assesses the usability of \acrshort{pgp}, \acrshort{smime} and \acrshort{pep} of a broad audience.
	\item We conduct \textbf{live observation tests}, in which we let participants use \acrshort{pgp}, \acrshort{smime} or \acrshort{pep} to write cryptographically securing emails. We observe the participants journey of installing, configuring and using the encryption features up to the point where we receive an email that was successfully encrypted and signed. We do these live observations to:
	\begin{itemize}
		\item validate the responses to the questions of our online survey.
		\item get a more reliable and precise feedback of the participants, revealing exactly which configuration step constitutes a challenge or which aspects hinder participants to apply email encryption in their regular mail exchange.
	\end{itemize}
\end{enumerate}
For the assessment of the three technologies, we divided it as following: one is responsible of reading the white-paper and two are responsible of configuring and testing the technology on every platform available. In same way for the live observations, we had twelve tests, each one of us took a number of tests and he was responsible to find a user to do it as well as supervise it.