%++++++++++++++++++++++++++++++++
%+   ABSTRACT                   +
%++++++++++++++++++++++++++++++++

%In this paper, we describe ....
% summary of work done in this / for this paper
In a world that is becoming more digitally interconnected every day, most
webservices running in the internet critically depend on cyber security measures for preventing cyber-crime and protecting user privacy. A huge step towards more secure internet communication has been the integration of end-to-end cryptography in mobile internet messenger services such as Whatsapp or Telegram. In contrast, for securing one of the most commonly used communication channels -- the \acrlong{email} -- end-to-end encryption is only applied to approximately 5\% of email traffic. That is even though end-to-end encryption technologies for \acrshort{email}s do exist since decades, namely \acrfull{pgp} and \acrfull{smime}. Our research analyses why users hestitate to secure their \acrshort{email} communication. Particularily we investigate which usability issues exist for \acrshort{pgp}, \acrshort{smime} as well as for a fairly new technology called \acrfull{pep}. Therefore we both execute an online survey and conduct live observations in which participants actively use encryption, in order to get a precise and authentic view on usability issues.
\newline
We found that more than 60\% of \acrshort{email} users are unaware of the existance of such encryption technologies and never tried to use one. We find that above all, users 1$)$ are overwelmed with the management of public keys and 2$)$ struggle with the setup of encryption technologie in their mail software. Even though users struggle to put \acrshort{email} end-to-end encryption into practice, we experience roughly the same amout of users being aware of the importance of \acrshort{email} encryption, with 66\% rating email encryption important or even very important. Additionally we find that users are very concerned about identity theft, as 78\% want to make sure that no other person is able to write email in their name. We conclude with improvements for each of the three encryption technologies analysed, that will help to reduce usability issues and in consequence hopefully lead to a wider adoption of end-to-end \acrshort{email} encryption.