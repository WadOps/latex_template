\section{Conclusion}
With our project, we identified the most frequent usability issues that users face when protecting their email communication using \acrlong{pgp} (\acrshort{pgp}), \acrlong{smime} (\acrshort{smime}) or \acrlong{pep} (\acrshort{pep}). First, we did a technical analysis of those three technologies, elaborating on trust establishment, key exchange, cryptographic algorithms and provided security features. 
Secondly we prepared and launched an online survey, aiming for a broad audience to collect a maximum feedback. The survey does not only identify the usability issues of each technology, but also assesses the general impression of our audience towards the importance of email encryption. Third, we conducted live observations in which participants were to install, configure and use either \acrshort{pgp}, \acrshort{smime} or \acrshort{pep} in presence of a team member that observed the participant testing the technologies and that provided support in case the participant encountered difficulties.
Our three fold approach gave us both, an overall view on the awareness of the email users as well as a detailed view on the causes for which users are hesitating to use the mentioned technologies. The overall impression we received from the online survey showed us that the email users are aware of the importance of email encryption with 32\% saying it's very important. Additionally, users are very concerned about identity theft, as 78\% of the participants want to make sure that no other person is able to write email using their name and 80\% of the participants want to be sure that the content of their mail isn't changed by someone else while being transferred to the recipient. It shows that for many users, signing emails is more important than encrypting them. For the usability issues we found, we propose some sample improvements that we suppose will make it easier for the users to apply \acrshort{pgp}, S/MIME or pEp in their email communication.
\newline
We present the improvement suggestions by technology:
\begin{itemize}
	\item \underline{Improvements for \acrshort{pgp}:}\newline
	We found out that on all tested platforms, users are restricted to the search for the recipient key only at one key server at a time, which makes importing the recipient key the major obstacle when using \acrshort{pgp}. Thus, we suggest to fix this issue by letting implementations search for the recipient key on all available key-servers at a time, without the user needing to manually adjust the key server for key import
	\item \underline{Improvements for \acrshort{smime}}\newline
	We suggest Certification Authorities to send further information on how to import the certificate into the most frequently used email programs, alongside the email which contains the requested digital certificate. Also, we suggest to integrate \acrshort{smime} support into webmail plugins, as webmail nowadays is used more commonly than dedicated desktop applications.
	\item \underline{Improvements for \acrshort{pep}}\newline
	We suggest \acrshort{pep} implementations to add further explanation on what to do with the trustwords displayed to the users during handshake. We furthermore advise to briefly explain the \acrshort{pep} color scheme that represents the security status of a communication channel to the user.
	Moreover we criticize the design choice of not being able to do a new handshake with a user whose public key was previously mistrusted. As a trustword mismatch can too easily occur due to a different language selection of the trustword dictionary on side of both users.	
\end{itemize}