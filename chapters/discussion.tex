\section{Discussion}
The objective of this part is to correlate the answers of the online survey with the results of live observations. We validate if the participants of our online survey criticize the same usability issues as the users in our live observations while using a certain technology. We extracted the following usability criteria so to compare the online survey with the live observations. These criteria are based on usability issues encountered during the online survey and the live observations (or both):
\begin{itemize}
	\item Compatibility
	\item Configuration
	\item Key Management
	\item Key pair generation
	\item Key import of other people’s public key
	\item Usage of an already existing key-pair
	\item Conditions to have fully secure communication channel
	\item Popularity of the technology
	\item Availability
	\item Number of steps to have a fully secure communication channel
\end{itemize}
We were able to validate every technology-specific answer from the online survey with the results of the live observations. Participants who answered technology-specific questions in our online survey have to be considered experienced using that respective technology. In constrast, most of our participants in the live observations never heard about either technologies before the test. However, we notice that both participant groups encounter the same issues, and as such strongly approve both independent results of our user study. For instance, the major obstacle for the participants of the online survey who stated to use \acrshort{pgp} is retrieving the recipient key (25\%). This showed to be the same major issue for the participants of our live observations. Moreover, the major obstacle of for the participants of the online survey who use \acrshort{smime} is to obtain a digital certificate from a trusted \acrlong{ca}, which also applies to our live observation participants. When we solved this problem for the participants of the live observations by pointing them towards a free certification service, the follow-up issue was configuring their environment to use \acrshort{smime} (i.e. import own certificate). This was also indicated by the online survey participants (17\%). Finally, we notice that the popularity of the three technologies is very low. We observe so from the online survey: 60\% never heard of \acrshort{pgp}, 64\% never heard of \acrshort{smime} and 90\% never
heard of \acrshort{pep}. Correspondingly, only one out of our 12 live observations participants heard of and also used \acrshort{pgp} (only one of three technologies). However, we believe that if the central usability issues carved out during this usability study are addressed by future releases of their implementations, user acceptance as well as the popularity of these technologies will increase. We propose improvements for each technology in chapter \ref{chap:conclusion}.
A detailed correlative analysis of our online survey dataset and our live observation dataset can be found for each technology independently in Table \ref{tab:SynthesePGP}, Table \ref{tab:SyntheseSMIME} and Table \ref{tab:SynthesepEp}

\begin{table*}[]
	\begin{tabular}{|l|l|}
		\hline
		{\textbf{Usability}}  	&\textbf{PGP} \\ \hline
		 {Compatibility.}		&\vtop{\hbox{\strut High compatibility:$\rightarrow$ OS: 68,4\%("No problem")}\hbox{\strut $\rightarrow$ vast support on many platforms (iOS, macOS, Win,Linux, Android, Webmail).}} \\ \hline
		 
		 {Configuration.}		&\vtop{\hbox{\strut Easy to integrate:OS: 5\%, "Mail clients have problems using PGP"}\hbox{\strut LO: Tested on all platforms without bugs}} \\ \hline
		 {Key Management.}		&Managed by the users.\\ \hline						
		 {Key pair generation.}	&LO: key creation difficult on Android \\ \hline	
		 {Key import of other people’s public key.} 	&\vtop{\hbox{\strut OS: 25\%, "It's difficult to to find other person's public key"}\hbox{\strut OS: 20\%, "Importing the keys of others consumes time"} \hbox{\strut LO: downloading from key server is time-consuming, yet keys are
		 		not always found}} \\ \hline
		 {Usage of an already existing key-pair.}		&A: Easy reuse the same key pair on different platforms.\\ \hline
		 	{\vtop{\hbox{\strut Conditions to have fully}\hbox{\strut secure communication channel.}}}	&\vtop{\hbox{\strut Search for the recipient public key:}\hbox{\strut $\rightarrow$ possibility to send the first mail encrypted and signed.}} \\ \hline
		 {Popularity of technology.}		&OS: 59.2\%, "Never heard of it".\\ \hline
		 {Availability.}		&\vtop {\hbox{Free on all platforms except iOS and macOS:}\hbox{ OS: 20\%, "I have to pay for the software (e.g Apple)"}\hbox{LO: Tested on the available platforms}} \\ \hline
		{\vtop{\hbox{\strut Number of steps to have a}\hbox{\strut fully secure communication channel}}}	&\vtop{\hbox {\strut LO: Seven steps (mobile).}\hbox{\strut 	Five steps (computer)}}\\ \hline
	\end{tabular}
	\caption{Correlation of online survey results with live observation results for \acrshort{pgp} \label{tab:SynthesePGP}}
	\vspace{5 pt}
		\small OS = Information obtained from the Online Survey. \\
		LO = Information obtained from the Live Observations. \\
		A = Information obtained from our Analysis.

\end{table*}

\begin{table*}[]
	\begin{tabular}{|l|l|}
		\hline
		{\textbf{Usability}}  	&\textbf{S/MIME} \\ \hline
		{Compatibility.}		&\vtop{\hbox{\strut High compatibility:$\rightarrow$ OS: 5,6\% ("my mail software does not support S/MIME")
				}\hbox{\strut $\rightarrow$ vast support on many platforms (iOS, macOS, Win,Linux, Android, Webmail).}\hbox{(attention: but not webmail, serious because of 50\% users us it)}} \\ \hline
		
		{Configuration.}		&\vtop{\hbox{\strut Easy to integrate OS: 16.7\%, "It's difficult to configure S/MIME on my environment"}\hbox{\strut LO: Settings aren't always easy to find to import the certificate.}} \\ \hline
		{Key Management.}		&OS: 27,8\%, "I don't trust digital certificates"\\ \hline						
		{Key pair generation.}	&\vtop {\hbox{OS: 11,1\%, "I don't know how to obtain a digital certificate"}\hbox{OS: 27,8\%, "I have to pay to obtain a digital certificate"}\hbox {LO: once found the service, it's easy to get certificate (keys)}} \\ \hline	
		{Key import of other people’s public key.} 	&\vtop{\hbox{OS: 27,8\%, "Yes", "Did you ever encounter ....untrusted certificate"}\hbox{A: certificate imported automatically when signed mail received}} \\ \hline
		{Usage of an already existing key-pair.}		&A: Easy to reuse the same certificate on different platforms\\ \hline
		{\vtop{\hbox{\strut Conditions to have fully}\hbox{\strut secure communication channel.}}}	&\vtop{\hbox{Send the first mail signed only to retrieve the recipient public key}\hbox{ $\rightarrow$ then send the next emails signed and encrypted.}} \\ \hline
		{Popularity of technology.}		&OS: 63.3\%, "Never heard of it".\\ \hline
		{Availability.}		&\vtop {\hbox{Free for 1 year on all platforms:}\hbox{OS: 27.8\%, "I have to pay to obtain a digital certificate}\hbox{LO: Tested on the available platforms}} \\ \hline
		{\vtop{\hbox{ Number of steps to have a}\hbox{ fully secure communication channel}}}	&\vtop{\hbox {LO: Eight steps (mobile).}\hbox{Five steps (computer).}}\\ \hline
	\end{tabular}
	\caption{Correlation of online survey results with live observation results for \acrshort{smime} \label{tab:SyntheseSMIME}}
	\vspace{5 pt}
	\small OS = Information obtained from the Online Survey. \\
		LO = Information obtained from the Live Observations. \\
		A = Information obtained from our Analysis.
\end{table*}

\begin{table*}[]
	\begin{tabular}{|l|l|}
		\hline
		{\textbf{Usability}}  	&\textbf{pEp} \\ \hline
		{Compatibility.}		&\vtop{\hbox{Low compatibility:$\rightarrow$ OS: 20\% ("my mail software does not support pEp")
			}\hbox{ $\rightarrow$ only Win, Linux, Android}\hbox{(attention: no webmail support !)}} \\ \hline
		
		{Configuration.}		&\vtop{\hbox{Hard to integrate:}\hbox{LO: Tested on only 2 platforms ("Android", "Thunderbird").}} \\ \hline
		{Key Management.}		&Managed automatically \\ \hline						
		{Key pair generation.}	&\vtop {\hbox{A: After installing the plugin, key pair generated automatically.}\hbox{LO: After installing the plugin, key pair generated automatically.}} \\ \hline	
		{Key import of other people’s public key.} 	&\vtop{\hbox{A: Abstraction, full automatic key management.}\hbox{$\rightarrow$ After receiving an email, it's done automatically.}\hbox{LO: After receiving an email, it's done automatically.}} \\ \hline
		{Usage of an already existing key-pair.}		&A: Bad key management. no synchronization between different platforms \\ \hline
		{\vtop{\hbox{\strut Conditions to have fully}\hbox{\strut secure communication channel.}}}	&\vtop{\hbox{Send the first email unencrypted and unsigned:}\hbox{ $\rightarrow$ Do the handshake to secure the channel.}\hbox{OS: 0\%, "Handshake fails often because trust words aren't the same"
		}} \\ \hline
		{Popularity of technology.}		&OS: 89.8\%, "Never heard of it".\\ \hline
		{Availability.}		&\vtop {\hbox{Free on Thunderbird only:}\hbox{OS: 20\%, "I have to pay for the software".}\hbox{LO: Tested on the available platforms}\hbox{A: the official version of the Android application.}} \\ \hline
		{\vtop{\hbox{ Number of steps to have a}\hbox{ fully secure communication channel}}}	&LO: Four steps.\\ \hline
	\end{tabular}
	\caption{Correlation of online survey results with live observation results for \acrshort{pep} \label{tab:SynthesepEp}}
	\vspace{5 pt}
	\small OS = Information obtained from the Online Survey. \\
		LO = Information obtained from the Live Observations. \\
		A = Information obtained from our Analysis.
\end{table*}

