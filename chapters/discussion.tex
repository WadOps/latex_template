\section{Discussion}
The objective of this part is to validate the answers of the online survey with the results of live observations. In other words, we try to find if the participants of the online survey have the same problems as the users in the live observations while using a certain technology and from that we can define precisely the hard steps that prevent the users to use one of the three technologies. We extracted the following usability criterias so we can compare the online survey with the live observations. In addition, these criteria were the usability challenges found either in the online survey, in the live observation or both. The usability criterias are:
\begin{itemize}
	\item Compatibility
	\item Configuration
	\item Key Management
	\item Key pair generation
	\item Key import of other people’s public key
	\item Usage of an already existing key-pair
	\item Conditions to have fully secure communication channel
	\item Popularity of the technology
	\item Availability
	\item Number of steps to have a fully secure communication channel
\end{itemize}
We were able to validate every answer in the the technologies sections we received from the online survey with the results of the live observations. On one hand, the participants of the online survey who answered the questions on the three technologies had experience using the technologies. On the other hand, the participants of the live observations never heard about the technologies before the test. We conclude that they had the same issues, as we can see from the results of online survey and the results of live observations. For example, the major obstacle
for the participants of the online survey who use \acrshort{pgp} is getting the recipient key (25\%) and it's the same issue for the participants of the live observations. Moreover, the major obstacle of for the participants of the online survey who use \acrshort{smime} is to obtain a free digital certificate from a trusted \acrlong{ca}. When we solved this problem for the participants of the live
observations, the following major issue was configuring their environment to use \acrshort{smime} (i.e. import own certificate), and that was the same issue for the participants from the online survey (17\%). Finally, we noticed that the popularity of three technologies is very low, as we observe from the online survey: 60\% never heard of \acrshort{pgp}, 64\% never heard of \acrshort{smime} and 90\% never
heard of \acrshort{pep}. In addition, one out of the twelve live observations, the user heard of \acrshort{pgp}, only one of three technologies. However, we believe that if we manage to solve all the difficulties in order to increase the usability, the popularity of the technologies will increase as well. So, we can accept the results and start to work on improvements for every technology in order to make
it more useable which will be discussed in the following section.