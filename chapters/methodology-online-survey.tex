\subsection{Online survey}
\textbf{\newline Goal of the online survey}
\newline
The aim of this online survey on email end-to-end encryption technologies was threefold. First, to explore users understanding and awareness of security in emails exchanges, their expectations and opinions on end-to-end encryption. Second, to establish a pattern about the propagation of the three technologies existing in the market. third, to compare this online survey which is quantitative data with the live survey defined as a qualitative data then validate those two surveys.
\newline
\newline  
\textbf{Structure}
\newline
Web based survey was our choice in order to collect data, because of its advantage, low cost and quick distribution. For the survey design, we tried to stimulate the participant to be objective, by keeping control on the flow of questions, for example, before asking specific question on each technology, we ask the participant if they have any knowledge of it. 
We tried to keep also an unbiased flow, by avoiding framing bias, through proper wording, and the use of clear, unambiguous and concise wording, in order to let the participant only depends on his personal knowledge on the topic.
The survey included closed-ended questions (multiple choice questions), open-ended questions and ranked questions with a balanced rating scale.
The online survey treats six sections, which one has its own purpose. Also, we included section skipping logic based on the participant answers.

\begin{itemize}
	\item The first section introduces the survey for the participant and handles the demographic data, but first it introduces the participant to the survey by explaining the aim of the survey, thenceforth it asks regular demographic questions and the type of the work organization to conclude if there is any relation between the need of secure email communications and the activity work type.
	\item The second section interact with the participant experience on the field of email exchange, this section helps us basically to identify the spread of email encryption through the participants email exchange, also make a link between the spread of email encryption and email clients usage.  
	\item The third section evaluate the participant experience with PGP, basically if the participant had already an experience with it, we discuss with him the advantage and the disadvantage of the technology related to its implementation on the email client based on our study on email client supporting end-to-end encryption.
	\item The fourth section evaluate the participant experience with S/MIME, mostly the same questions as the previous section adjusted for the technology.
	The fifth section evaluate the participant experience with pep, this section is shorter than the previous ones by reason of being new to the market, it includes some of the previous questions also adjusted to the technology.
	\item The last and sixth section, gather the overall impression on end-to-end encryption, by scaling the degree of awareness of the participants on matter of the security of email exchange especially if they had an experience with an email piracy issue.
\end{itemize}