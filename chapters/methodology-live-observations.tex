\subsection{Live observations}
\textbf{\newline Analysis of commonly used \acrlong{mua}}
\newline
We test the integration of \acrshort{pgp}, \acrshort{smime} and \acrshort{pep} in today's most commonly used mail programs (\acrshort{mua}) ourselves that support end-to-end encryption, in order to
\begin{itemize}
	\item identify which encryption technology is supported by which \acrshort{mua}, as well as to prevent participants from testing \acrshort{mua}s that turn out to be unusable (e.g. due to discontinued development, incompatibility of versions and operating systems ...)
	\item anticipate which kind of challenges users will face when trying to use these three technologies in the context of a specific \acrshort{mua}
	\item be able to assess challenges faced by users and to help them overcome common pitfalls that would otherwise ultimately hinder them from sending us a secure \acrshort{email}.
\end{itemize}
Table \ref{tab:MailUserAgents} represents the \acrlong{mua}s which we considered in our analysis. It also depicts the plugin which is necessary to add \acrshort{pgp}, \acrshort{smime} or \acrshort{pep} functionality to a \acrshort{mua} if not	supported natively.
{\def\arraystretch{1.2}\tabcolsep=3pt
\begin{table*}%[!h]
	\scriptsize
	\centering
	\begin{tabular}{|c|l|l|c|}
		\hline
		\textbf{Technology}		& \textbf{\acrlong{mua}}   		& \textbf{Plugin}  				& \textbf{Tested}	\\ \hline
								& Outlook Desktop 2013/2016		& Gpg4o							& \cmark	\\ \cline{2-4}
								& Thunderbird					& Enigmail 						& \cmark 	\\ \cline{2-4}
								& Gmail(Webmail)				& FlowCrypt						& \cmark	\\ \cline{2-4}
		{PGP}					& Other Webmail  				& Mailvelope            		& \cmark	\\ \cline{2-4}
								& Apple iOS						& iPG mail app.         		& \xmark	\\ \cline{2-4}
								& Android						& Maildroid and Cryptoplugin	& \cmark	\\ \cline{2-4}
		     					& Windows Mail 					& Not Supported					& \xmark	\\ \cline{2-4}
		     					& MacOS							& Not Supported					& \xmark 	\\ \hline
		     					
								& Outlook Desktop 2013/2016		& Native support				& \cmark	\\ \cline{2-4}
								& Thunderbird					& Native support 				& \cmark 	\\ \cline{2-4}
								& Apple iOS						& iPhone mail app				& \cmark	\\ \cline{2-4}
		{S/MIME}				& Android						& Maildroid and Cryptoplugin    & \cmark	\\ \cline{2-4}
								& MacOS							& Native support         		& \cmark	\\ \cline{2-4}
								& Gmail(Webmail)				& Not Supported 				& \xmark	\\ \cline{2-4}
								& Other Webmail 				& Not Supported					& \xmark	\\ \cline{2-4}
								& Windows Mail					& Native support				& \xmark 	\\ \hline
								
								& Thunderbird					& Enigmail						& \cmark	\\ \cline{2-4}
								& Android						& Official pEp app				& \cmark 	\\ \cline{2-4}
								& Apple iOS						& App coming soon				& \xmark	\\ \cline{2-4}
		{pEp}					& Outlook Desktop 2013/2016		& pEp for outlook    			& \xmark	\\ \cline{2-4}
								& MacOS							& Not supported         		& \xmark	\\ \cline{2-4}
								& Gmail(Webmail)				& Not Supported 				& \xmark	\\ \cline{2-4}
								& Other Webmail 				& Not Supported					& \xmark	\\ \cline{2-4}
								& Windows Mail					& Not supported					& \xmark 	\\ \hline 	
	\end{tabular}
	\caption{Commonly used mail user agents (MUA) and their support for \acrshort{pgp}, \acrshort{smime} and \acrshort{pep}} \label{tab:MailUserAgents}
\end{table*}
}
From this collection of \acrshort{mua}s we had to find a subset of \acrshort{mua}s that we would use for our live observations. We did this selection, because our project’s principal contribution is an assessment of the usability of \acrshort{pgp}, \acrshort{smime} and \acrshort{pep} and not the general usability of a broad variety of mail programs. The chosen subset considers the popularity of the \acrshort{mua}s as well as it considers that each encryption technology should be tested on each major platform (if supported) and should be free of costs for our participants. We assume that even if the technology is usable, its implementation within an \acrshort{email} program might make it difficult to use and vice-versa. Therefore we wanted our participants to test two
different implementations of \acrshort{email} encryption, to have a direct comparison of their usability:
\newline
A participant would either test two different technologies or he would test two different implementations of the same technology. Particularly, when testing two different technologies, we let participants test a \acrshort{pep} implementation and a \acrshort{pgp} or \acrshort{smime} implementation, in order to see if \acrshort{pep} indeed meets its goal of simplifying the mail encryption process.
In total we did 24 live observations with 12 different participants, and each test was supervised by one team member of of our project working group.
\newline
\newline
\textbf{Live observation testing protocol}
\newline
Each live test was conducted adhering to a predefined testing protocol. Each live test starts with a short interview of the participant, determining some demographic data (age, nationality, profession), the prefered \acrshort{mua} to access his \acrshort{email}s, and if and which previous knowledge the participant has about cryptography in general or email encryption in particular. In case the participant has experience in using one of the \acrshort{mua}s among those we decided on using for the live testing, he is assigned to a respective test scenario. We do so to reduce the complexity for the participant, meaning that he would be concerned with configuring and using encryption features rather than struggling with an unknown mail software. For the same reason, we helped participants installing and setting up a \acrshort{mua} up to the point that they are able to successfully access their mail account, in case they didn’t have previous experience with any \acrshort{mua} of our list. The participant is then asked to enable and configure the security features of his \acrshort{mua} to use a specific mail encryption technology and send a secured mail to the project working group. Very few information is provided to the participant by the supervisor. If the participant is struggling for more than 10 minutes with a specific configuration step, he received help from the supervisor. The live test is completed once the supervisor received an \acrshort{email} by the participant that was successfully encrypted and signed.