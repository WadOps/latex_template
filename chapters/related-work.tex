\section{Related Work}
In this section, we will give brief overview on related work.\\
In 2012, Moecke and Volkamer analyzed all different email services,defining security, usability, interoperability criteria and apply them to existing approaches. Based on the result, closed and web-based systems like Hushmail are more usable, contrarily to PGP/SMIME, which require add-ons to carry the key in a secure way.\cite{usable-secure-email}\\
In 2017, Lerner, Zeng and Roesner from University of Washington, presented a case study with people who frequently conduct sensitive business, they estimate confidence put on encrypted emails using a prototype they developed based on Keybase for automatic key management.\cite{confidente}\\
In 2018, Clark, van Oorschot, Ruoti, Seamons and Zappala conducted a study focused on: Systematization of secure email approaches taken by industry, academia, and independent developers; An evaluation for proposed or deployed email security enhancements and a measurement of their security, deployability, and usability. Through their study, they concluded that Deployment and adoption of end-to-end encrypted email continues to face many challenges: Usability on a day-to-day scale; Key management which remains very unpractical.\cite{secure-email}\\
In 2018, A group of researchers from Brigham Young University and University of Tennessee conducted a comparative usability study of key management in secure emails tools, in which they oversee a comparative based study between passwords, public key directory (PKD), and identity-based encryption (IBE), the result of the study demonstrates that each key management has its potential to be successfully used in secure email \cite{comparativestudy}.


