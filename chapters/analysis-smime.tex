\subsection{\acrfull{smime}}
\label{chap:analysis-smime}

With the increase of power in terms of computation and networking capability, people's need for using non-textual objects such as image, audio, and video in emails became a fair demand. To support diversity of content, multipart message structure, and non-English text, the Multipurpose Internet Mail Extension (MIME) was proposed in 1993 \cite{rfc1521}. S/MIME (Secure/Multipurpose Internet Mail Extension) is an enhancement of MIME to provide cryptographic security for the MIME based emails \cite{rfc1521}. To better understand S/MIME, let us first take look at MIME.
S/MIME adds some additional content types to the MIME to provide security services. The current S/MIME version 3.2 obsoletes all earlier versions. However, most implementations still bear version 3.1 features for digital signature processing. Today, popular email clients such as MS Outlook 2013, MS Outlook 2016, Mozilla Thunderbird, Apple Mail support S/MIME enabled messages.
S/MIME version 3.2 is defined by the following five major specifications.
\begin{itemize}
\item Cryptographic Message Syntax (RFC 4853 \cite{rfc4853})
\item Cryptographic Message Syntax (CMS) Algorithms (RFC 3370 \cite{rfc3370})
\item Diffie-Hellman Key Agreement Method (RFC 2631 \cite{rfc2631})
\item S/MIME Version 3.2 Certificate Handling (RFC 5750 \cite{rfc5750})
\item S/MIME Version 3.2 Message Specification (RFC 5751 \cite{rfc5751})
\item Enhanced Security Services for S/MIME (RFC 2634 \cite{rfc2634})
\end{itemize}
\textbf{Cryptographic Message Syntax (CMS)}
To define how security services, such as confidentiality or integrity, can be added to MIME content types, S/MIME defines Cryptographic Message Syntax (CMS). The syntax in each case defines the exact encoding scheme for each content type. Discussed below are the different types of messages and different sub-types that are created from these messages \cite{PKCS-7}.
\begin{itemize}
\item Data Content Type is an arbitrary string.  The object created is called "Data".
\item Digested-Data Content Type is used to provide integrity for the message. The result is typically used as the content for the enveloped-data content type.  The encoded result is an object called "digested-data".  The process of creating "digested-data" involves the following two steps
\begin{enumerate}
\item Using the hash algorithm of the user's choice a message digest is created from the content.
\item The message digest, the algorithm, and the content are added together to create the "digested-data" object.
\end{enumerate}
\item Signed Data Content Type provides authenticity and integrity of data. It contains any type and zero or more signature values. The encoded result is an object called "signed-data". A signed data message can only be viewed by the recipient of with S/MIME capability. The following are the steps in the process.
\begin{enumerate}
\item For each signer, a message digest is created from the content using the specific hash algorithm chosen by the signer.
\item Each message digest is signed by the signer with his or her private key.
\item The content, signature values, certificates, and the algorithms are then collected to create the "signed-data".
\end{enumerate}
\item Enveloped-Data Content Type is used to provide privacy for the message. It contains encrypted content of any type, and zero or more encrypted keys and certificates. The encoded result is an object called "enveloped-data". The steps involved in the process is are as follows
\begin{enumerate}
\item A pseudo-random session key is created for the symmetric-key algorithm to be used to encrypt the content.
\item The content is encrypted using the defined algorithm and created session key.
\item For each recipient, a copy of the session key is encrypted with the public key of the recipient.
\item The encrypted contents, encrypted session keys, algorithm used, and certificates are encoded using Radix-64.
\end{enumerate}
\end{itemize}
\textbf{Cryptographic Algorithms}
S/MIME specifies several cryptographic algorithms for use. The term "must" indicate absolute requirement, and the term "should" imply recommendation only. S/MIME recommends using SHA-256 as hash function for creating message digests. For content encryption RSA is recommended.
\newline
\textbf{Key Management and Certification}
The key management in S/MIME is a combination of key management used by X.509 and PGP. S/MIME uses public key certificates signed by the Certification Authorities (CA) defined by X.509. The user is responsible for maintaining the web of trust to verify signatures as defined by PGP. In general, for key establishment, a user needs to get his or her pair of public and private keys certified by trusted CAs \cite{SEmailETU}.
